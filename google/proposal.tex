
The prisoner's dilemma is elegant in its simplicity: two prisoners---denied communication---must decide whether to cooperate with each other or defect.  This foundational thought experiment of game theory has helped understand many real world scenarios: \dots .  Despite its power, the prisoner's dilemma is woefully unrealistic.  Cooperation and betrayal do not happen in a cell cut off from the rest of the world.  Instead, it is mediated by communication: promises are made, then broken, and met with recriminations.  

We propose to study the role of \emph{language} in understanding betrayal and cooperation in the game of {\bf Diplomacy}.  Diplomacy, like the prisoner's dilemma, is a repeated game where players choose to either cooperate or betray other players.  Unlike the prisoner's dilemma, however, a key component of Diplomacy is \emph{communication} to convince another player to help you and betray your foes.

Another key difference is that Diplomacy is fun and played around the world.  